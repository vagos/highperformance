%----------------------------------------------------------------------------------------
%	PACKAGES AND OTHER DOCUMENT CONFIGURATIONS
%----------------------------------------------------------------------------------------

\documentclass[11pt]{scrartcl} % Font size

\input{structure.tex} % Include the file specifying the document structure and custom commands
\usepackage{multirow}
\usepackage{array}
% Define a macro to create a table with fixed column widths
\newcolumntype{C}[1]{>{\centering\arraybackslash}p{#1}}

\usepackage{hyperref}
\hypersetup{
    colorlinks=true,
    linkcolor=blue,
    filecolor=magenta,      
    urlcolor=cyan,
}

%----------------------------------------------------------------------------------------
%	TITLE SECTION
%----------------------------------------------------------------------------------------

\title{	
	\normalfont\normalsize
	\textsc{Πανεπιστήμιο Πατρών, Τμήμα Μηχανικών ΗΥ και Πληροφορικής}\\ % Your university, school and/or department name(s)
	\vspace{25pt} % Whitespace
	\rule{\linewidth}{0.5pt}\\ % Thin top horizontal rule
	\vspace{20pt} % Whitespace
	{\LARGE Λογισμικό και Προγραμματισμός Συστημάτων Υψηλής Επίδοσης\\ Άσκηση 2 - PCA και SIMD}\\ % The assignment title
	\vspace{12pt} % Whitespace
	\rule{\linewidth}{2pt}\\ % Thick bottom horizontal rule
	\vspace{12pt} % Whitespace
}


\author{Ευάγγελος Λάμπρου \\UP1066519 \and Ιωάννης Παναρίτης \\UP1072632} % Your name

\date{} % Today's date (\today) or a custom date

%----------------------------------------------------------------------------------------
%	DOCUMENT
%----------------------------------------------------------------------------------------

\bibliographystyle{abbrv}
\addto\captionsgreek{\renewcommand{\refname}{Αναφορές}}

\begin{document}

\maketitle 

\section{Principal Component Analysis}

\subsection{Υλοποίηση}


\subsection{Μετρήσεις}

\begin{table}[!ht]
    \centering
    \begin{tabular}{|l|l|}
    \hline
        \textbf{k} & \textbf{compression} \\ \hline
        1 & 127.793 \\ \hline
        30 & 9.00192 \\ \hline
        50 & 5.48538 \\ \hline
        100 & 2.77515 \\ \hline
    \end{tabular}
\end{table}

\begin{table}[!ht]
    \centering
    \begin{tabular}{|l|c|c|c|c|c|c|}
    \hline
        \textbf{Image} & \multicolumn{6}{|c|}{\textbf{Time Elapsed (s)}} \\
    \hline
        & Mean/STD & Normal & C-Matrix & \src{dsyev} & PC-Reduced & \textbf{Overall} \\ \hline
        \textbf{elvis}             & 0.0006647         & 0.0034834       & 0.199793          & 0.269401       & 0.0158582           & 0.490011         \\ \hline
        \textbf{cyclone}           & 0.0300084         & 0.13506         & 62.3818           & 65.7206        & 1.17117             & 129.439          \\ \hline
        \textbf{earth}             & 0.165224          & 0.72195         & 772.333           & 826.566        & 12.1875             & 1594.48          \\ \hline

        \textbf{elvis\_parallel}   & 0.0030338         & 0.0037285       & 0.0293645         & 0.26617        & 0.0175036           & 0.3225           \\ \hline
        \textbf{cyclone\_parallel} & 0.0063717         & 0.183004        & 15.7129           & 64.3138        & 1.17857             & 81.396           \\ \hline
        \textbf{earth\_parallel}   & 0.0252138         & 0.99866         & 213.427           & 840.459        & 12.1471             & 1067.06          \\ \hline
    \end{tabular}
\end{table}


\subsection{Έλεγχος Αποτελεσμάτων}


\section{Forces with SIMD and OpenMP}

\subsection{Υλοποίηση}

Ξεκινώντας με την αρχική υλοποίηση, μεταφράσαμε τις πράξεις για τον υπολογισμό της δύναμης μεταξύ δύο σωματιδίων σε εντολές SIMD 
οι οποίες θα υπολογίζουν σε κάθε επανάληψη τη δύναμη μεταξύ ενός σωματιδίου με 4 άλλα σωματίδια ταυτόχρονα. 
Εκμεταλευόμαστε το ότι η δύναμη μεταξύ δύο σωματιδίων δεν εξαρτάται από αυτή που ασκείται μεταξύ δύο άλλων (αρχή επαλληλίας).

Χρησιμοποιήσαμε SIMD intrinsics για την AVX οικογένεια επεξεργαστών βασιζόμενοι στο \href{https://www.intel.com/content/www/us/en/docs/intrinsics-guide/index.html#text=mm256_loadu_pd&ig_expand=4488&techs=AVX_ALL}{documentation που προσφέρει η Intel}.

Τελικά, αφού μεταφράσαμε τον αρχικό κώδικα έτσι ώστε να υπολογίζεται για κάθε σωματίδιο η δύναμη που του ασκείται
από άλλα τέσσερα, παραλληλοποιήσαμε τον υπολογισμό της συνολικής δύναμης για το σύνολο των σωματιδίων χρησιμοποιώντας 
OpenMP directives.

\subsection{Μετρήσεις}

Οι παρακάτω μετρήσεις έγιναν σε σύστημα με επεξεργαστή \src{Intel(R) Core(TM) i3-2310M CPU @ 2.10GHz}.

\subsubsection{\src{vanilla} Υλοποίηση}

\begin{table}[!ht]
    \centering
    \begin{tabular}{|c|c|}
    \hline
        \textbf{Particles} & \textbf{Time Elapsed (s)} \\ \hline
        3     & 1.1e-05  \\ \hline 
        32    & 0.000101 \\ \hline
        2017  & 0.277711 \\ \hline
        11111 & 8.47835  \\ \hline
    \end{tabular}
\end{table}

\subsubsection{\src{fast} Υλοποίηση (SIMD Εντολές) (Χωρίς OpenMP)}

\begin{table}[!ht]
    \centering
    \begin{tabular}{|c|c|}
    \hline
        \textbf{Particles} & \textbf{Time Elapsed (s)} \\ \hline
            3     & 3e-06    \\ \hline
            32    & 6.8e-05  \\ \hline
            2017  & 0.16086  \\ \hline
            11111 & 4.64873  \\ \hline
    \end{tabular}
\end{table}

\subsubsection{\src{fast} Υλοποίηση (SIMD Εντολές και OpenMP)}

\begin{table}[!ht]
    \centering
    \begin{tabular}{|c|c|}
    \hline
        \textbf{Particles} & \textbf{Time Elapsed (s)} \\ \hline
            3     & 0.000934 \\ \hline
            32    & 9e-06    \\ \hline
            2017  & 0.022682 \\ \hline
            11111 & 0.671763 \\ \hline
    \end{tabular}
\end{table}

Βλέπουμε πως η τελική υλοποίηση έχει τα καλύτερα αποτέσματα για μεγάλα μεγέθη εισόδου.
Στην περίπτωση των τριών (3) σωματιδίων, η τρίτη υλοποίηση είναι πιο αργή, πράγμα κατανοητό αν λάβουμε υπόψιν μας το χρόνο αρχικοποίησης των νημάτων και άλλων επιβαρύνσεων.

\subsection{Έλεγχος Αποτελεσμάτων}

\bibliography{bibliography}

\end{document}
