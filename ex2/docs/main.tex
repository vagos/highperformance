%----------------------------------------------------------------------------------------
%	PACKAGES AND OTHER DOCUMENT CONFIGURATIONS
%----------------------------------------------------------------------------------------

\documentclass[11pt]{scrartcl} % Font size

%%%%%%%%%%%%%%%%%%%%%%%%%%%%%%%%%%%%%%%%%
% Wenneker Assignment
% Structure Specification File
% Version 2.0 (12/1/2019)
%
% This template originates from:
% http://www.LaTeXTemplates.com
%
% Authors:
% Vel (vel@LaTeXTemplates.com)
% Frits Wenneker
%
% License:
% CC BY-NC-SA 3.0 (http://creativecommons.org/licenses/by-nc-sa/3.0/)
% 
%%%%%%%%%%%%%%%%%%%%%%%%%%%%%%%%%%%%%%%%%

%----------------------------------------------------------------------------------------
%	PACKAGES AND OTHER DOCUMENT CONFIGURATIONS
%----------------------------------------------------------------------------------------

\usepackage{amsmath, amsfonts, amsthm} % Math packages

\usepackage{listings} % Code listings, with syntax highlighting

\usepackage{graphicx} % Required for inserting images

\usepackage{booktabs} % Required for better horizontal rules in tables

\usepackage{dirtytalk} % Required for quoting.

\usepackage{float} % Added for hard placement of images.

\usepackage[dvipsnames]{xcolor} % Added for extra colors.

\usepackage{tikz} % For colored boxes and more.

\numberwithin{equation}{section} % Number equations within sections (i.e. 1.1, 1.2, 2.1, 2.2 instead of 1, 2, 3, 4)
\numberwithin{figure}{section} % Number figures within sections (i.e. 1.1, 1.2, 2.1, 2.2 instead of 1, 2, 3, 4)
\numberwithin{table}{section} % Number tables within sections (i.e. 1.1, 1.2, 2.1, 2.2 instead of 1, 2, 3, 4)

\usepackage{enumitem} % Required for list customisation
\setlist{noitemsep} % No spacing between list items

\usepackage[main=greek, english]{babel}

%----------------------------------------------------------------------------------------
%	DOCUMENT MARGINS
%----------------------------------------------------------------------------------------

\usepackage{geometry} % Required for adjusting page dimensions and margins

\geometry{
	paper=a4paper, % Paper size, change to letterpaper for US letter size
	top=2.5cm, % Top margin
	bottom=3cm, % Bottom margin
	left=3cm, % Left margin
	right=3cm, % Right margin
	headheight=0.75cm, % Header height
	footskip=1.5cm, % Space from the bottom margin to the baseline of the footer
	headsep=0.75cm, % Space from the top margin to the baseline of the header
	%showframe, % Uncomment to show how the type block is set on the page
}

%----------------------------------------------------------------------------------------
%	FONTS
%----------------------------------------------------------------------------------------

\usepackage[utf8]{inputenc} % Required for inputting international characters
\usepackage[T1]{fontenc} % Output font encoding for international characters

\usepackage{XCharter} % Use the XCharter fonts


%----------------------------------------------------------------------------------------
%	SECTION TITLES
%----------------------------------------------------------------------------------------

\usepackage{sectsty} % Allows customising section commands

\sectionfont{\vspace{6pt}\centering\normalfont\scshape} % \section{} styling
\subsectionfont{\normalfont\bfseries} % \subsection{} styling
\subsubsectionfont{\normalfont\itshape} % \subsubsection{} styling
\paragraphfont{\normalfont\scshape} % \paragraph{} styling

%----------------------------------------------------------------------------------------
%	HEADERS AND FOOTERS
%----------------------------------------------------------------------------------------

\usepackage{scrlayer-scrpage} % Required for customising headers and footers

\ohead*{} % Right header
\ihead*{} % Left header
\chead*{} % Centre header

\ofoot*{} % Right footer
\ifoot*{} % Left footer
\cfoot*{\pagemark} % Centre footer

\newcommand{\img}[1] % maybe add a caption to this
{
    \begin{center}
        \fcolorbox{black}{white}{\includegraphics[width=\textwidth]{#1}}
    \end{center}

}

% Helper Macros

\newcommand{\en}[1]{\foreignlanguage{english}{#1}}
\newcommand{\src}[1]{{\texttt{#1}}}


% Extra Formatting

\setlength{\parindent}{0em}
\setlength{\parskip}{0em}


% Code Listing Style

\lstdefinestyle{code}{
  belowcaptionskip=1\baselineskip,
  breaklines=true,
  frame=LRTB,
  xleftmargin=\parindent,
  showstringspaces=false,
  basicstyle=\ttfamily,
  keywordstyle=\bfseries\color{green!40!black},
  commentstyle=\itshape\color{purple!40!black},
  identifierstyle=\color{black},
  stringstyle=\color{orange},
}


\newcommand{\lstcode}[3]
{
    \begin{otherlanguage}{english}
    \lstinputlisting[language=#2, frame=single, style=code, caption=#3]{#1}
    \end{otherlanguage}
}
 % Include the file specifying the document structure and custom commands
\usepackage{multirow}
\usepackage{array}
\usepackage{subcaption}

% Define a macro to create a table with fixed column widths
\newcolumntype{C}[1]{>{\centering\arraybackslash}p{#1}}

\usepackage{hyperref}
\hypersetup{
    colorlinks=true,
    linkcolor=blue,
    filecolor=magenta,      
    urlcolor=cyan,
}


\newcommand{\showpcaimage}[1]{
    \begin{figure}[H]
        \centering
        \begin{subfigure}[b]{0.45\textwidth}
            \includegraphics[width=\textwidth]{./assets/#1_1.jpg}
            \caption{Ποσοστό από Principal Components -- 1\%}
        \end{subfigure}
        \begin{subfigure}[b]{0.45\textwidth}
            \includegraphics[width=\textwidth]{./assets/#1_25.jpg}
            \caption{Ποσοστό από Principal Components -- 25\%}
        \end{subfigure}
        \begin{subfigure}[b]{0.45\textwidth}
            \includegraphics[width=\textwidth]{./assets/#1_75.jpg}
            \caption{Ποσοστό από Principal Components -- 75\%}
        \end{subfigure}
        \begin{subfigure}[b]{0.45\textwidth}
            \includegraphics[width=\textwidth]{./assets/#1_100.jpg}
            \caption{Ποσοστό από Principal Components -- 100\%}
        \end{subfigure}
        \caption{Ποσοστό από Principal Components που κρατήσαμε. Εικόνα \src{#1.jpg}}
    \end{figure}
}

%----------------------------------------------------------------------------------------
%	TITLE SECTION
%----------------------------------------------------------------------------------------

\title{	
	\normalfont\normalsize
	\textsc{Πανεπιστήμιο Πατρών, Τμήμα Μηχανικών ΗΥ και Πληροφορικής}\\ % Your university, school and/or department name(s)
	\vspace{25pt} % Whitespace
	\rule{\linewidth}{0.5pt}\\ % Thin top horizontal rule
	\vspace{20pt} % Whitespace
	{\LARGE Λογισμικό και Προγραμματισμός Συστημάτων Υψηλής Επίδοσης\\ Άσκηση 2 - PCA και SIMD}\\ % The assignment title
	\vspace{12pt} % Whitespace
	\rule{\linewidth}{2pt}\\ % Thick bottom horizontal rule
	\vspace{12pt} % Whitespace
}


\author{Ευάγγελος Λάμπρου \\UP1066519 \and Ιωάννης Παναρίτης \\UP1072632} % Your name

\date{} % Today's date (\today) or a custom date

%----------------------------------------------------------------------------------------
%	DOCUMENT
%----------------------------------------------------------------------------------------

\bibliographystyle{abbrv}
\addto\captionsgreek{\renewcommand{\refname}{Αναφορές}}

\begin{document}

\maketitle 

\section{Principal Component Analysis}

\subsection{Υλοποίηση}

Για την υλοποίηση αυτής της άσκησης έγινε \say{μετάφραση} του κώδικα Matlab σε C++.
Βασική διαφορά είναι η χρήση της βιβλιοθήκης Lapack για τις πράξεις γραμμικής άλγεβρας (υπολογισμός ιδιοτιμών/ιδιοδιανυσμάτων, πολλαπλασιασμός πινάκων).

Με OMP έχουμε παραλληλοποιήσει τα βήματα του PCA όπως: 

\begin{itemize}
    \item Υπολογισμός mean/std 
    \item Normalization
    \item Υπολογισμός του Covariance Matrix.
    \item Επιλογή των $k$ Principal Components.
\end{itemize}

\subsection{Μετρήσεις}

\begin{table}[H]
    \centering
    \begin{tabular}{|l|l|}
    \hline
        \textbf{k} & \textbf{compression} \\ \hline
        1 & 127.793 \\ \hline
        30 & 9.00192 \\ \hline
        50 & 5.48538 \\ \hline
        100 & 2.77515 \\ \hline
    \end{tabular}
    \caption{Το ποσοστό συμπίεσης για διαφορετικό αριθμό principal components που κρατήσαμε για την εικόνα \src{elvis}.}
\end{table}

\begin{table}[H]
    \centering
    \begin{tabular}{|l|c|c|c|c|c|c|}
    \hline
        \textbf{Image} & \multicolumn{6}{|c|}{\textbf{Time Elapsed (s)}} \\
    \hline
        & Mean/STD & Normal & C-Matrix & \src{dsyev} & PC-Reduced & \textbf{Overall} \\ \hline
        \textbf{elvis}             & 0.0006647         & 0.0034834       & 0.199793          & 0.269401       & 0.0158582           & 0.490011         \\ \hline
        \textbf{cyclone}           & 0.0300084         & 0.13506         & 62.3818           & 65.7206        & 1.17117             & 129.439          \\ \hline
        \textbf{earth}             & 0.165224          & 0.72195         & 772.333           & 826.566        & 12.1875             & 1594.48          \\ \hline

        \textbf{elvis\_parallel}   & 0.0030338         & 0.0037285       & 0.0293645         & 0.26617        & 0.0175036           & 0.3225           \\ \hline
        \textbf{cyclone\_parallel} & 0.0063717         & 0.183004        & 15.7129           & 64.3138        & 1.17857             & 81.396           \\ \hline
        \textbf{earth\_parallel}   & 0.0252138         & 0.99866         & 213.427           & 840.459        & 12.1471             & 1067.06          \\ \hline
    \end{tabular}

    \caption{Ο χρόνος εκτέλεσης για τα διάφορα βήματα του PCA. Οι μετρήσεις είναι για όλες τις εικόνες χρησιμοποιώντας την σειριακή και την παράλληλη έκδοση του προγράμματος.}
\end{table}


\subsection{Έλεγχος Αποτελεσμάτων}

Τα αποτελέσματα παραγωγής των εικόνων είναι ως εξής. 
Κάθε φορά κρατάμε ένα ποσοστό των ιδιοτιμών/ιδιοδυανυσμάτων για την κάθε εικόνα (1\%, 50\%, 75\%, 100\%). 

Παρατηρήσαμε ότι το αποτέλεσμα είναι διαφορετικό από αυτό όταν τρέχουμε τον κώδικα Matlab.
Αυτό πιθανότατα συμβαίνει λόγω της διαφορετικής υλοποίησης του αλγορίθμου υπολογισμού των ιδιοτιμών
στη Matlab και στη βιβλιοθήκη Lapack. 
Η Matlab χρησιμποιεί τον αλγόριθμο Jacobi ο οποίος είναι πιο αργός αλλά πιο ακριβής από τον αλγόριθμο QR που χρησιμοποιεί 
η συνάρτηση \src{dsyev} της Lapack.

\showpcaimage{elvis}

\section{Forces with SIMD and OpenMP}

\subsection{Υλοποίηση}

Ξεκινώντας με την αρχική υλοποίηση, μεταφράσαμε τις πράξεις για τον υπολογισμό της δύναμης μεταξύ δύο σωματιδίων σε εντολές SIMD 
οι οποίες θα υπολογίζουν σε κάθε επανάληψη τη δύναμη μεταξύ ενός σωματιδίου με 4 άλλα σωματίδια ταυτόχρονα. 
Εκμεταλευόμαστε το ότι η δύναμη μεταξύ δύο σωματιδίων δεν εξαρτάται από αυτή που ασκείται μεταξύ δύο άλλων (αρχή επαλληλίας).

Χρησιμοποιήσαμε SIMD intrinsics για την AVX οικογένεια επεξεργαστών βασιζόμενοι στο \href{https://www.intel.com/content/www/us/en/docs/intrinsics-guide/index.html#text=mm256_loadu_pd&ig_expand=4488&techs=AVX_ALL}{documentation που προσφέρει η Intel}.

Τελικά, αφού μεταφράσαμε τον αρχικό κώδικα έτσι ώστε να υπολογίζεται για κάθε σωματίδιο η δύναμη που του ασκείται
από άλλα τέσσερα, παραλληλοποιήσαμε τον υπολογισμό της συνολικής δύναμης για το σύνολο των σωματιδίων χρησιμοποιώντας 
OpenMP directives.

\subsection{Μετρήσεις}

Οι παρακάτω μετρήσεις έγιναν σε σύστημα με επεξεργαστή \src{Intel(R) Core(TM) i3-2310M CPU @ 2.10GHz}.

\subsubsection{\src{vanilla} Υλοποίηση}

\begin{table}[H]
    \centering
    \begin{tabular}{|c|c|}
    \hline
        \textbf{Particles} & \textbf{Time Elapsed (s)} \\ \hline
        3     & 1.1e-05  \\ \hline 
        32    & 0.000101 \\ \hline
        2017  & 0.277711 \\ \hline
        11111 & 8.47835  \\ \hline
    \end{tabular}
\end{table}

\subsubsection{\src{fast} Υλοποίηση (SIMD Εντολές) (Χωρίς OpenMP)}

\begin{table}[H]
    \centering
    \begin{tabular}{|c|c|}
    \hline
        \textbf{Particles} & \textbf{Time Elapsed (s)} \\ \hline
            3     & 3e-06    \\ \hline
            32    & 6.8e-05  \\ \hline
            2017  & 0.16086  \\ \hline
            11111 & 4.64873  \\ \hline
    \end{tabular}
\end{table}

\subsubsection{\src{fast} Υλοποίηση (SIMD Εντολές και OpenMP)}

\begin{table}[H]
    \centering
    \begin{tabular}{|c|c|}
    \hline
        \textbf{Particles} & \textbf{Time Elapsed (s)} \\ \hline
            3     & 0.000934 \\ \hline
            32    & 9e-06    \\ \hline
            2017  & 0.022682 \\ \hline
            11111 & 0.671763 \\ \hline
    \end{tabular}
\end{table}

Βλέπουμε πως η τελική υλοποίηση έχει τα καλύτερα αποτλέσματα για μεγάλα μεγέθη εισόδου.
Στην περίπτωση των τριών (3) σωματιδίων, η τρίτη υλοποίηση είναι πιο αργή, πράγμα κατανοητό αν λάβουμε υπόψιν μας το χρόνο αρχικοποίησης των νημάτων και άλλων επιβαρύνσεων.

\subsection{Έλεγχος Αποτελεσμάτων}

Ο έλεγχος αποτελεσμάτων γίνεται αυτόματα με βάση τον υπολογισμό των δυνάμεων από τον σειριακό κώδικα.
Χρησιμοποιήσαμε το πρόγραμμα \src{generate} για τη δημιουργία των πρότυπων τιμών.

{\ttfamily
\lstinputlisting{assets/passed.txt}
}

% \bibliography{bibliography}

\end{document}
