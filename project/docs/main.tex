%----------------------------------------------------------------------------------------
%	PACKAGES AND OTHER DOCUMENT CONFIGURATIONS
%----------------------------------------------------------------------------------------

\documentclass[11pt]{scrartcl} % Font size

\input{structure.tex} % Include the file specifying the document structure and custom commands
\usepackage{multirow}
\usepackage{array}
\usepackage{subcaption}
% Bar chart drawing library 
\usepackage{pgfplots} 
\usepackage{textcomp}
\usepackage{algorithm}
\usepackage{algpseudocode}

% Define a macro to create a table with fixed column widths
\newcolumntype{C}[1]{>{\centering\arraybackslash}p{#1}}

\usepackage{hyperref}
\hypersetup{
    colorlinks=true,
    linkcolor=blue,
    filecolor=magenta,      
    urlcolor=cyan,
}

\definecolor{codegreen}{rgb}{0,0.6,0}
\definecolor{codegray}{rgb}{0.5,0.5,0.5}
\definecolor{codepurple}{rgb}{0.58,0,0.82}
\definecolor{backcolour}{rgb}{0.95,0.95,0.92}
\definecolor{codeblue}{rgb}{0,0,0.8}

\lstdefinestyle{mystyle}{
    backgroundcolor=\color{backcolour},   
    commentstyle=\color{codegreen},
    keywordstyle=\color{codeblue},
    numberstyle=\tiny\color{codegray},
    stringstyle=\color{codepurple},
    basicstyle=\ttfamily\footnotesize,
    breakatwhitespace=false,         
    breaklines=true,                 
    captionpos=b,                    
    keepspaces=true,                 
    numbers=left,                    
    numbersep=5pt,                  
    showspaces=false,                
    showstringspaces=false,
    showtabs=false,                  
    tabsize=2
}

\lstset{style=mystyle}

%----------------------------------------------------------------------------------------
%	TITLE SECTION
%----------------------------------------------------------------------------------------

\title{	
	\normalfont\normalsize
	\textsc{Πανεπιστήμιο Πατρών, Τμήμα Μηχανικών ΗΥ και Πληροφορικής}\\ % Your university, school and/or department name(s)
	\vspace{25pt} % Whitespace
	\rule{\linewidth}{0.5pt}\\ % Thin top horizontal rule
	\vspace{20pt} % Whitespace
    {\Large Λογισμικό και Προγραμματισμός Συστημάτων Υψηλής Επίδοσης \\ \textbf{Final Project:} Function approximation with k-Nearest Neighbors}\\ % The assignment title
	\vspace{12pt} % Whitespace
	\rule{\linewidth}{2pt}\\ % Thick bottom horizontal rule
	\vspace{12pt} % Whitespace
}

\author{Ευάγγελος Λάμπρου \\UP1066519 \and Ιωάννης Παναρίτης \\UP1072632} % Your name

\date{} % Today's date (\today) or a custom date

%----------------------------------------------------------------------------------------
%	DOCUMENT
%----------------------------------------------------------------------------------------

\bibliographystyle{abbrv}
\addto\captionsgreek{\renewcommand{\refname}{Αναφορές}}


\begin{document}

\maketitle 

% \tableofcontents

\section{Υλοποιήσεις}

\subsection{Σειριακή}
\subsection{OpenMP}
\subsection{OpenACC}
\subsection{CUDA}
\subsection{MPI}
\subsubsection{MPI Cluster}

% setup ssh
% setup nfs
% change /etc/exports
% exportfs -a
% restart nfs
% from client: sudo mount -t nfs -o master:/home/vagozino/mpi
% run RUN_CMD="mpirun -np 4 --host laptop:0,desktop:4 ./diffusion2d_mpi $D $L $N $T $DT" 

% mpi 1 machine: 4.307 time/query 20.35ms 
% mpi 3 machines: 26.753 time/query 12.54ms

% create a bar plot for the total time for 1 machine and 3 machines
% create a bar plot for the time/query for 1 machine and 3 machines
\begin{figure}[H]
    \begin{center}
\begin{tikzpicture}
    \begin{axis}[
        xlabel={Αριθμός Μηχανημάτων},
        ylabel={Time},
        xtick=data,
        xticklabels={1, 3},
        ybar,
        enlarge x limits=0.15,
        bar width=10pt,
        legend style={at={(0.5,-0.20)}, anchor=north, legend columns=-1},
        legend entries={Total Time (s), Time/query (ms)},
        ymajorgrids=true,
        y grid style=dashed,
        nodes near coords,
        nodes near coords align={vertical},
        nodes near coords style={anchor=north},
        every node near coord/.append style={yshift=-2pt},
        every axis plot post/.append style={fill=blue!20},
        ]
        \addplot coordinates {(1, 4.307) (3, 26.753)};
        \addplot coordinates {(1, 20.35) (3, 12.54)};
    \end{axis}
\end{tikzpicture}
    \end{center}
    \caption{Χρόνοι εκτέλεσης της υλοποίηση με και χωρίς CUDA.}
    \label{fig:cluster_times}
\end{figure}




\section{Μετρήσεις}

\section{Έλεγχος Αποτελεσμάτων}

\bibliography{bibliography}

\end{document}
