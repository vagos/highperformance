%----------------------------------------------------------------------------------------
%	PACKAGES AND OTHER DOCUMENT CONFIGURATIONS
%----------------------------------------------------------------------------------------

\documentclass[11pt]{scrartcl} % Font size

\input{structure.tex} % Include the file specifying the document structure and custom commands

%----------------------------------------------------------------------------------------
%	TITLE SECTION
%----------------------------------------------------------------------------------------

\title{	
	\normalfont\normalsize
	\textsc{Πανεπιστήμιο Πατρών, Τμήμα Μηχανικών ΗΥ και Πληροφορικής}\\ % Your university, school and/or department name(s)
	\vspace{25pt} % Whitespace
	\rule{\linewidth}{0.5pt}\\ % Thin top horizontal rule
	\vspace{20pt} % Whitespace
	{\LARGE Λογισμικό και Προγραμματισμός Συστημάτων Υψηλής Επίδοσης\\ Άσκηση 1 - MPI και ΜΔΕ}\\ % The assignment title
	\vspace{12pt} % Whitespace
	\rule{\linewidth}{2pt}\\ % Thick bottom horizontal rule
	\vspace{12pt} % Whitespace
}


\author{Ευάγγελος Λάμπρου \and Ιωάννης Παναρίτης} % Your name

\date{} % Today's date (\today) or a custom date

%----------------------------------------------------------------------------------------
%	DOCUMENT
%----------------------------------------------------------------------------------------

\bibliographystyle{abbrv}
\addto\captionsgreek{\renewcommand{\refname}{Αναφορές}}

\begin{document}

\maketitle 

\section{2D Diffusion και MPI}

    \subsection{Υλοποίηση}
        Επιλέχθηκε το blocking implementation για την υλοποίησή.\\
        Πραγματοποιήθηκε αλλαγή στις συναρτήσεις \src{init(), initialize\_density(), advance(),\\compute\_diagnostics()}.

        Για την επικοινωνία μεταξύ των διεργασιών χωρίζουμε ολόκληρο το grid σε τετράγωνα, με την κάθε διεργασία να χειρίζεται ένα από αυτά.
        Η επικοινωνία μεταξύ των διεργασιών γίνεται με την ανταλλαγή των γειτονικών στηλών και γραμμών. \cite{ghostcellpattern}

        \begin{figure}[htpb]
            \centering
            \includegraphics[width=0.5\textwidth]{./assets/ghostcells.png}
            \caption{Η ανταλλαγή ghost cell στηλών και γραμμών μεταξύ των διεργασιών.}
        \end{figure}
        
        \subsubsection*{Αρχικές μετατροπές}
            Προσθέτουμε στο \src{struct Diffusion2D}:
            \begin{itemize}
                \item \src{Local\_N, Square\_N} ώστε να γνωρίζουμε το μέγεθος του κάθε τετραγώνου
                \item Buffer για send, receive με τους γείτονές του.
            \end{itemize}
            
        \subsubsection*{\src{Init()}}
            \begin{itemize}
                 \item Υπολογισμός \src{grid\_size}, για να ξέρουμε σε πόσα μέλη χωρίζεται η κάθε πλευρά.
                \item Υπολογισμός του \src{square\_N} και των υπολοιπών τιμών που χρειαζόμαστε.
                \item Δέσμευση χώρου για \src{right, left} ανταλλαγή δεδομένων.
            \end{itemize}
        \subsubsection*{\src{Initialize\_density()}}
            \begin{itemize}
                \item Χρήση \src{Local\_N\_} αντί για \src{N\_}
                \item Υπολογισμός \src{gj} και χρήση του αντί για \src{j}
            \end{itemize}
        \subsubsection*{\src{Advance()}}
            \begin{itemize}
                \item Χρήση \src{Local\_N\_} αντί για \src{N\_}
                \item Υπολογισμός \src{id} των γειτόνων \src{(left, right, up, down)}
                \item Send και Receive δεδομένων με τους γείτονες
            \end{itemize}
        \subsubsection*{\src{compute\_diagnostics()}}
        \begin{itemize}
            \item Χρήση \src{Local\_N\_} αντί για \src{N\_}
        \end{itemize}

    \subsection{Μετρήσεις}

    Οι μετρήσεις έγιναν σε σύστημα με χαρακτηριστικά: 

    \begin{itemize}
        \item Processor: AMD Ryzen 7 2700 Eight-Core Processor 3.20 GHz
        \item RAM: 16.0 GB
        \item Operating System: 64-bit Linux
    \end{itemize}

    \begin{center}
        \begin{tabular}{|p{3cm}||p{3cm}|p{3cm}|}
            \hline
            \multicolumn{3}{|c|}{Execution Time ($s$)} \\
            \hline
            \src{N} & 1 process & 4 processes\\
            \hline
            1024 & 1.677881 & 0.741665\\
            2048 & 6.756217 & 4.03658\\
            4096 & 26.015809 & 15.681488\\
            \hline
        \end{tabular}
    \end{center}

    Στη συνέχεια προσθέσαμε παραλληλισμό με threads για την κάθε διεργασία στην επανάληψη για τον υπολογισμό του νέου πίνακα \src{rho}. Τα αποτελέσματα είναι τα εξής:
    
    \begin{center}
        \begin{tabular}{|p{3cm}||p{3cm}|p{3cm}|}
            \hline
            \multicolumn{3}{|c|}{Execution Time ($s$)} \\
            \hline
            \src{N} & 1 process & 4 processes\\
            \hline
            1024 &    0.521331 &    4.260742 \\
            2048 &    4.017462 &    5.913613 \\
            4096 &    15.52263 &    16.333386 \\
            \hline
        \end{tabular}
    \end{center}

    Φαίνεται πως στην περίπτωση των 4 διεργασιών ο χρόνος εκτέλεσης αυξάνεται.
    Πιθανότατα η παραληλοποίηση σε επίπεδο νημάτων να είναι ανούσια εφόσων το σύστημα δεν έχει αρκετούς πυρήνες για την εκτέλεση του παράλληλου κώδικα.
    Έτσι, το παράλληλο κομμάτι παραμένει ουσιαστικά συριακό, έχoντας όμως το overhead του συγχρονισμού των νημάτων.
    Ενδεχομένως η λύση αυτή σε ένα μηχάνημα με περισσότερους πυρήνες να ήταν καλύτερη.
        
\section{Chekpointing με MPI I/O}

    \subsection{Υλοποίηση}

    Για την υλοποίηση χρησιμοποιήσαμε του μηχανισμούς I/O που προσφέρει η βιβλιοθήκη MPI.

    Για να αποφύγουμε την αποθήκευση των ghost cell γραμμών, μεταφέρουμε τα
    non-ghost cells πρώτα στον ενδιάμεσο buffer \src{file\_buffer\_} πριν τον κάνουμε
    serialize τελικά στο αρχείο.

        \subsubsection*{\src{write\_density\_mpi()}}
            \begin{itemize}
                \item Άνοιγμα αρχείου για \src{write}
                \item Υπολογισμός \src{length} και \src{offset}
                \item \src{file\_buffer\_ $\leftarrow $ rho}
                \item Αποθήκευση του πίνακα \src{file\_buffer\_} στο αρχείο χρησιμοποιώοντας την blocking κλήση κλήση \src{MPI\_File\_write}
            \end{itemize}
        \subsubsection*{\src{read\_density\_mpi()}}
            \begin{itemize}
                \item Άνοιγμα αρχείου για \src{read}
                \item Υπολογισμός \src{length} και \src{offset}
                \item Δίαβασμα \src{len doubles} από \src{offset} και αποθήκευση στο \src{file\_buffer\_} χρησιμοποιώντας την blocking κλήση κλήση \src{MPI\_File\_read}.
                \item \src{rho $\leftarrow $ file\_buffer\_}
            \end{itemize}


    \subsection{Έλεγχος Αποτελεσμάτων}

        Τρέχουμε πρόγραμμα για μία και τέσσερις διεργασίες.
        Γίνεται \src{break} στο ίδιο σημείο και συγκρίνουμε τα binary αρχεία (\src{.restart} αρχεία) που προκύπτουν.

        \begin{verbatim}
            mpiio$ ./run.sh # Run the code for 1 process
            timestep from stability condition is 2.388849e-07
            Timing: 1024 0.031531
            mpiio$ mv density_mpi.dat.restart density_mpi_1.dat.restart
            mpiio$ ./run.sh # Run the code for 4 processes
            timestep from stability condition is 2.388849e-07
            Timing: 1024 0.011161
            mpiio$ mv density_mpi.dat.restart density_mpi_4.dat.restart
            mpiio$ diff density_mpi_1.dat.restart density_mpi_4.dat.restart
        \end{verbatim}
        
        Η εντολή \src{diff} επιστρέφει κενό, άρα δεν υπάρχει κάποια διαφορά στα αρχεία.

\bibliography{bibliography}
    
\end{document}
