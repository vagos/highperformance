%----------------------------------------------------------------------------------------
%	PACKAGES AND OTHER DOCUMENT CONFIGURATIONS
%----------------------------------------------------------------------------------------

\documentclass[11pt]{scrartcl} % Font size

\input{structure.tex} % Include the file specifying the document structure and custom commands

%----------------------------------------------------------------------------------------
%	TITLE SECTION
%----------------------------------------------------------------------------------------

\title{	
	\normalfont\normalsize
	\textsc{Πανεπιστήμιο Πατρών, Τμήμα Μηχανικών ΗΥ και Πληροφορικής}\\ % Your university, school and/or department name(s)
	\vspace{25pt} % Whitespace
	\rule{\linewidth}{0.5pt}\\ % Thin top horizontal rule
	\vspace{20pt} % Whitespace
	{\LARGE Λογισμικό και Προγραμματισμός Συστημάτων Υψηλής Επίδοσης\\ Άσκηση 1 - MPI και ΜΔΕ}\\ % The assignment title
	\vspace{12pt} % Whitespace
	\rule{\linewidth}{2pt}\\ % Thick bottom horizontal rule
	\vspace{12pt} % Whitespace
}


\author{Ευάγγελος Λάμπρου \and Ιωάννης Παναρίτης} % Your name

\date{} % Today's date (\today) or a custom date

%----------------------------------------------------------------------------------------
%	DOCUMENT
%----------------------------------------------------------------------------------------

\bibliographystyle{abbrv}
\addto\captionsgreek{\renewcommand{\refname}{Αναφορές}}

\begin{document}

\maketitle 

\section{2D Diffusion και MPI}

    \subsection{Υλοποίηση}
        Επιλέχθηκε το blocking implementation για την υλοποίησή.\\
        Πραγματοποιήθηκε αλλαγή στις συναρτήσεις \src{init(), initialize\_density(), advance(),\\compute\_diagnostics()}.

        Για την επικοινωνία μεταξύ των διεργασιών χωρίζουμε ολόκληρο το grid σε τετράγωνα, με την κάθε διεργασία να χειρίζεται ένα από αυτά.
        Η επικοινωνία μεταξύ των διεργασιών γίνεται με την ανταλλαγή των γειτονικών στηλών και γραμμών. \cite{ghostcellpattern}

        \begin{figure}[htpb]
            \centering
            \includegraphics[width=0.5\textwidth]{./assets/ghostcells.png}
            \caption{Η ανταλλαγή ghost cell στηλών και γραμμών μεταξύ των διεργασιών.}
        \end{figure}
        
        \subsubsection*{Αρχικές μετατροπές}
            Προσθέτουμε στο \src{struct Diffusion2D}:
            \begin{itemize}
                \item \src{Local\_N, Square\_N} ώστε να γνωρίζουμε το μέγεθος του κάθε τετραγώνου
                \item Buffer για send, receive με τους γείτονές του.
            \end{itemize}
            
        \subsubsection*{\src{Init()}}
            \begin{itemize}
                 \item Υπολογισμός \src{grid\_size}, για να ξέρουμε σε πόσα μέλη χωρίζεται η κάθε πλευρά.
                \item Υπολογισμός του \src{square\_N} και των υπολοιπών τιμών που χρειαζόμαστε.
                \item Δέσμευση χώρου για \src{right, left} ανταλλαγή δεδομένων.
            \end{itemize}
        \subsubsection*{\src{Initialize\_density()}}
            \begin{itemize}
                \item Χρήση \src{Local\_N\_} αντί για \src{N\_}
                \item Υπολογισμός \src{gj} και χρήση του αντί για \src{j}
            \end{itemize}
        \subsubsection*{\src{Advance()}}
            \begin{itemize}
                \item Χρήση \src{Local\_N\_} αντί για \src{N\_}
                \item Υπολογισμός \src{id} των γειτόνων \src{(left, right, up, down)}
                \item Send και Receive δεδομένων με τους γείτονες
            \end{itemize}
        \subsubsection*{\src{compute\_diagnostics()}}
        \begin{itemize}
            \item Χρήση \src{Local\_N\_} αντί για \src{N\_}
        \end{itemize}

    \subsection{Μετρήσεις}

    Οι μετρήσεις έγιναν σε σύστημα με χαρακτηριστικά: 

    \begin{itemize}
        \item Processor: AMD Ryzen 7 2700 Eight-Core Processor 3.20 GHz
        \item RAM: 16.0 GB
        \item Operating System: 64-bit Linux
    \end{itemize}
    
    \centering
    \begin{tabular}{|p{3cm}||p{3cm}|p{3cm}|}
        \hline
        \multicolumn{3}{|c|}{Execution Time ($s$)} \\
        \hline
        \src{N} & 1 process & 4 processes\\
        \hline
        1024 & 1.677881 & 0.741665\\
        2048 & 6.756217 & 4.03658\\
        4096 & 26.015809 & 15.681488\\
        \hline
    \end{tabular}
        
\section{Chekpointing με MPI I/O} 

\bibliography{bibliography}
    
\end{document}
