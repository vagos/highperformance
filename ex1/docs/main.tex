%----------------------------------------------------------------------------------------
%	PACKAGES AND OTHER DOCUMENT CONFIGURATIONS
%----------------------------------------------------------------------------------------

\documentclass[11pt]{scrartcl} % Font size

\input{structure.tex} % Include the file specifying the document structure and custom commands

%----------------------------------------------------------------------------------------
%	TITLE SECTION
%----------------------------------------------------------------------------------------

\title{	
	\normalfont\normalsize
	\textsc{Πανεπιστήμιο Πατρών, Τμήμα Ηλεκτρολόγων Μηχανικών και Τεχνολογίας Υπολογιστών}\\ % Your university, school and/or department name(s)
	\vspace{25pt} % Whitespace
	\rule{\linewidth}{0.5pt}\\ % Thin top horizontal rule
	\vspace{20pt} % Whitespace
	{\huge Τίτλος Εργασίας}\\ % The assignment title
	\vspace{12pt} % Whitespace
	\rule{\linewidth}{2pt}\\ % Thick bottom horizontal rule
	\vspace{12pt} % Whitespace
}

\author{Ευάγγελος Λάμπρου \and Ιωάννης Παναρίτης} % Your name

\date{} % Today's date (\today) or a custom date

%----------------------------------------------------------------------------------------
%	DOCUMENT
%----------------------------------------------------------------------------------------

\begin{document}

\maketitle 

\section*{Question 1}

    \subsection*{Μερικά λόγια για την υλοποίηση}
        Επιλέχθηκε το blocking implementation για την υλοποίησή.\\
        Πραγματοποιήθηκε αλλαγή στις συναρτήσεις \texttt{init(), initialize\_density(), advance(),\\compute\_diagnostics()}.
        
        \subsubsection*{Αρχικές μετατροπές}
            Προσθέτουμε στο \texttt{struct Diffusion2D}:
            \begin{itemize}
                \item \texttt{Local\_N, Square\_N} ώστε να γνωρίζουμε το μέγεθος του κάθε τετραγώνου
                \item Buffer για send, receive με τους γείτονές του.
            \end{itemize}
            
        \subsubsection*{\texttt{Init()}}
            \begin{itemize}
                 \item Υπολογισμός \texttt{grid\_size}, για να ξέρουμε σε πόσα μέλη χωρίζεται η κάθε πλευρά.
                \item Υπολογισμός του \texttt{square\_N} και των υπολοιπών τιμών που χρειαζόμαστε.
                \item Δέσμευση χώρου για \texttt{right, left} ανταλλαγή δεδομένων.
            \end{itemize}
        \subsubsection*{\texttt{Initialize\_density()}}
            \begin{itemize}
                \item Χρήση \texttt{Local\_N\_} αντί για \texttt{N\_}
                \item Υπολογισμός \texttt{gj} και χρήση του αντί για \texttt{j}
            \end{itemize}
        \subsubsection*{\texttt{Advance()}}
            \begin{itemize}
                \item Χρήση \texttt{Local\_N\_} αντί για \texttt{N\_}
                \item Υπολογισμός \texttt{id} των γειτόνων \texttt{(left, right, up, down)}
                \item Send και Receive δεδομένων με τους γείτονες
            \end{itemize}
        \subsubsection*{\texttt{compute\_diagnostics()}}
        \begin{itemize}
            \item Χρήση \texttt{Local\_N\_} αντί για \texttt{N\_}
        \end{itemize}
    
    \subsection*{Μετρήσεις}
        \texttt{ /* TO DO */}
            
        

\end{document}
